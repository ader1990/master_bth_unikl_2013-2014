\documentclass[conference]{IEEEtran}
\usepackage{cite}
\usepackage{epigraph}

% \epigraphsize{\small}% Default
\setlength\epigraphwidth{8cm}
\setlength\epigraphrule{0pt}

\usepackage{etoolbox}

\makeatletter
\patchcmd{\epigraph}{\@epitext{#1}}{\itshape\@epitext{#1}}{}{}
\makeatother

\ifCLASSINFOpdf
\else
\fi

% correct bad hyphenation here
\hyphenation{leadership technology although considered}

\begin{document}
%
% paper title
% can use linebreaks \\ within to get better formatting as desired
% Do not put math or special symbols in the title.
\title{Leadership for a mixed team  of \newline junior and senior developers}


% author names and affiliations
% use a multiple column layout for up to three different
% affiliations
\author{\IEEEauthorblockN{Adrian Vladu}
\IEEEauthorblockA{European Master in Software Engineering\\
Blekinge Institute of Technology, \\
Karlskrona, Sweden\\
Email: adrian.vladu21@gmail.com}}

% make the title area
\maketitle

% As a general rule, do not put math, special symbols or citations
% in the abstract
\begin{abstract}
This paper describes the issues which arrise from building a team of developers
with highly different experience level. The leader has to address the challenge by using the appropriate 
mix of knowledge, processes and tools. 
\newline
Considering the multitude of existent variables, the leader should prioritize the organization vision and connect it to the creative, innovative and often diverging energies of the team members. The emphasys of this paper is on the prevalence of human relations over tools and processes. The negative effects of experience imbalance on the team behaviour can be reduced or redirected to a positive result, primarily relying on effective communication and motivators. Leadership is done by humans, not tools.
\end{abstract}

\IEEEpeerreviewmaketitle

\section{Task description}
\textit{Imagine yourself as a project manager of a group of eight developers.
From these eight, four developers are experienced and mature, while four are young university graduates who have been just recruited. 
\newline\indent 
Describe the effects of imbalance in the experience on the dynamics of team behaviour, and your role as the leader in addressing this challenge.}

\section{Introduction}
A leader's activity is based on uncertainty. One person, on a given time and situation, cannot be aware of all the factors, thus creating the incentive for a less than rational behaviour, which is hard to predict. On the other hand, a leader who has experience and a multitude of tools, processes and insights into bahavioural options that followers can choose from, will exhibit a higher rate of success.   
\newline\indent
In the situation presented, the antagonism between experience and maturity versus the inexperience and youth will cause a disruption in how the two different groups perceive themselves and the environment. As a consequence, the expected actions in the same situation will differ. The role of the leader is to find solutions of combining the different reactions in a productive and less conflict prone way. The leadership process should prove itself successful on the long-term, rather than on the short-term.
\newline\indent
Whilst the material resources are of relatively low importance, as those can be easily created and substituted, the human resources are paramount for an organization, taking years to train, form or replace.    

\section{Factors and actors}    
\emph{David W. Galenson}, in his work \emph{``Old Masters and Young Geniuses: The Two Life Cycles of Artistic Creativity"}, although he refers to plastic artists, I strongly believe that simmilar ideas can be applied to programming and software engineering. Consequently, software products can be considered as the unique result of an art form. The team members' creativity and past knowledge can balance each other. This will happen if the development process is managed properly, with the leader acting as a middleman between the two molding forces - innovation and reuse.
\newline\indent
The ability to achieve a goal, to finish successfully a task can be native or hard-earned. One may innovate successfully at the conceptual level, but it requires a long time of trial-and-error or learning for experimental and solid innovation. The ability to radically innovate can decrease with age, as a person is inclined to self-establish a set of rigid methods and conventions.\cite{IEEEhowto:oldvsyoung}

\epigraph{``When a situation requires a new way of looking at things, the acquisition of new techniques, or even new vocabularies, the old seem stereotyped and rigid....But when a situation requires a store of past knowledge then the old find their advantage over the young."}{--- \textup{Harvey Lehman}, 1953
}
For a leader, the most important assets are the members of the group. A thorough understanding of the humans is of crucial importance, as all the predictions should have a solid starting ground. To accomplish the first desiderate, depending on the situations that may occur, there are specific methods that can be used to gather information about the members. First of all, the data can be collected using non-interactive methods like analyzing the history in the company, previous results, references, Curriculum Vitae, social and cultural activity.
\newline\indent
Secondly, a more indepth analysis can be done using active methods, such as an interview, informal discussions. The profiling should be a continous process, as people change over time. According to the iceberg theory of Sigmund Freud\cite{IEEEhowto:freud}, the largest part of our mind is concealed and very hard to analyze on normal circumstances. The knowledge of the unconscious mind is a great asset, as it significantly influences behaviour. If the process is successful, the current and future actions of the team will be easier to understand and predict.
Other important factors are the organization's culture, demographics, vision and current political, geographical and social environment. The multitude of factors should be properly adressed and understood by the leader, so that the future decisions would be in full concordance with them.  

\section{Building a team}
First of all, the members of the team should get in contact with each other, so that they have a starting point to develop work relations. The interaction should be encouraged and stimulated on breaks, while working or facilitated during more formal or informal activities like teambuildings, nights out, attending the opera or theater. All of the above activities should not be compulsory and made as natural as possible, in order for the members to accommodate at their own pace.
\newline\indent
Secondly, the implementation of agile methodologies are considered of great use in this scenario. Agile gives the leader a framework focused on continous communication and collaboration, which should easily be used to level up the distance between the team members. Formal daily stand-ups and weekly one hour meetings give the team members the occasion to speak, interact, share and change ideas, seek and give help. But a hammer won't always do the job.\cite{IEEEhowto:oldvsyoung} The Agile methodology could be inappropriate for some projects or situations, although the activities presented above could be implemented while following any project life cycle model.
\newline\indent
As for software development, external trainings should be organized, held by highly skilled professionals outside the team, but also internal trainings, with team members as trainers. The young will give trainings about newly appeared technology topics, which they can choose on the condition those are related to the project's scope. I consider that the best way to learn something is to teach someone. Also, giving the young developers the oportunity, their self-esteem and self-trust will increase. In parallel, the senior developers will give trainings on the subjects they master perfectly, so that the young developers get more insight about the used technology and know whom to ask if they have a certain issue. The motivation for the seniors is the opportunity to present to the team their high level skills, gaining respect and acknowledgement. 
\newline\indent
Once every couple of weeks, teambuildings should be organized, with the purpose to strengthen the connections and also having the chance to know eachothers' social, moral values.
\newline\indent
During the first three or four weeks of practical software development, shadowing can be a very efficient way for the juniors to rapidly acquire knowledge. One junior will stay with one senior and just observe what the senior is doing. Every week, the junior will change to another senior. The choice will be made by the team. After the shadowing, there will be a change to pear programming, the senior and the junior taking places while programming. The juniors' responsabilities will increase gradually, and after the pair programming, he will become relatively independent and have his own tasks. Nonetheless, a junior can anytime ask for help from a senior. After a period of six to nine months, it will be a decrease in the maximum performance of the team, as the seniors are involved also in coaching the juniors, but after nine to then weeks, the performance should start to rise, as the juniors are able to handle their own tasks. 
\newline\indent
The initial momentum and excitement for the new endeavour of the team members should be taken advantage of, and the team should be given responsabilities from the start. The fact that the juniors get a mentor they can rely on will satisfy their security needs, so that they can perform without being distracted by the possible negative consequences they work can have over the software system.
For the juniors, they will be given the chance to break things and then try to put the together again. It will inspire the juniors and they will believe more in their capabilities and power.
\newline\indent
As technology rapidly advances, and today's methods can become obsolete in a relative short period of time, a twenty percent slack time is offered to the team members for their own projects, curiosities and endeavours. The slack time will start to show positive results if the members of the team work together or use it to learn something new. By giving the members the freedom to choose their very own goals it will make them more responsible and motivated. People can respond in responsible and productive ways to a work environment in which they are given an opportunity to grow and mature. People begin to satisfy their esteem and self-actualization needs by participating in the planning, organizing, motivating and controlling of their own tasks.(page 72)
\newline\indent
On the other hand, there has to be acknowledged and address the situation of unwanted behavior, of undesired unofficial groups being created or alliances between the team members of same age. In the same time, people are social and gregarious beings, so group formation must not be interdicted or suppressed, only controlled and directed to meet the organization's vision and project goals.
\newline\indent
Other issue that may be encountered is that the juniors may have less personal constraints(children, marriage, long-time relationships in private life) than the seniors. Also, seniors may be reluctant and not so willing to spend their entire creative energy on job-related issues. The family may be, for them, on the first plan. The issue can be addressed using communication techniques between the leader and the members.
\newpage

\section{Conclusion}
The problem that was addressed in this paper has no straightforward solutions, but the tactics and techniques presented should give to the leader the required mix of tools, processes and knowledge in order to enhance his performance. By leveraging the knowledge about team members and implementing the proposed processes, the outcome of the should have a high level of predictability.  

% use section* for acknowledgement
\section*{Acknowledgment}
The author would like to thank, first of all, to his teachers and, secondly, to the high availability of energizers. Without the existence of the two elements, this paper could not have been successfully completed. 




\newpage
\begin{thebibliography}{99}


\bibitem{IEEEhowto:fulltime}
Paul Hersey, Kenneth H. Blanchard and Dewey E. Johnson, \emph{Management of Organizational Behavior: Leading Human Resources}

\bibitem{IEEEhowto:oldvsyoung}
David W. Galenson, \emph{Old Masters and Young Geniuses: The Two Life Cycles of Artistic Creativity}

\bibitem{IEEEhowto:freud}
Freud, S. (1920). \emph{Beyond the pleasure principle. SE, 18: 1-64.}

\bibitem{IEEEhowto:lehman}
\emph{Harvey Lehman, 1953} 

\end{thebibliography}

\end{document}


