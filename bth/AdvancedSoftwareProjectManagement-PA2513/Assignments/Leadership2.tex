\documentclass[conference]{IEEEtran}
\usepackage{cite}
\usepackage{epigraph}

% \epigraphsize{\small}% Default
\setlength\epigraphwidth{8cm}
\setlength\epigraphrule{0pt}

\usepackage{etoolbox}

\makeatletter
\patchcmd{\epigraph}{\@epitext{#1}}{\itshape\@epitext{#1}}{}{}
\makeatother
% *** GRAPHICS RELATED PACKAGES ***
%
\ifCLASSINFOpdf
  % \usepackage[pdftex]{graphicx}
  % declare the path(s) where your graphic files are
  % \graphicspath{{../pdf/}{../jpeg/}}
  % and their extensions so you won't have to specify these with
  % every instance of \includegraphics
  % \DeclareGraphicsExtensions{.pdf,.jpeg,.png}
\else
  % or other class option (dvipsone, dvipdf, if not using dvips). graphicx
  % will default to the driver specified in the system graphics.cfg if no
  % driver is specified.
  % \usepackage[dvips]{graphicx}
  % declare the path(s) where your graphic files are
  % \graphicspath{{../eps/}}
  % and their extensions so you won't have to specify these with
  % every instance of \includegraphics
  % \DeclareGraphicsExtensions{.eps}
\fi

% correct bad hyphenation here
\hyphenation{op-tical net-works semi-conduc-tor leader-ship technology}

\begin{document}
%
% paper title
% can use linebreaks \\ within to get better formatting as desired
% Do not put math or special symbols in the title.
\title{Leadership for a mixed team of junior and senior developers}


% author names and affiliations
% use a multiple column layout for up to three different
% affiliations
\author{\IEEEauthorblockN{Adrian Vladu}
\IEEEauthorblockA{European Master in \\Software Engineering\\
Blekinge Institute of Technology\\
Karlskrona, Sweden\\
Email: adrian.vladu21@gmail.com}}

% make the title area
\maketitle

% As a general rule, do not put math, special symbols or citations
% in the abstract
\begin{abstract}
Leadership is done by humans, not tools.
\end{abstract}

\IEEEpeerreviewmaketitle

%\section{Task}
%In this assignment you are required to write an essay to demonstrate your understanding of the leadership role in software projects and ability to apply it in the given situation. Detailed requirements follow.
%\newline\indent
%Imagine yourself as a project manager of a group of eight developers.
%From these eight, four developers are experienced and mature, while four are young university graduates who have been just recruited.
%\newline\indent
%Describe the effects of imbalance in the experience on the dynamics of team behaviour, and your role as the leader in addressing this challenge. 
%\newline

\section{Techniques used}
Analyze each team member, using the available formal or informal information: history in the company, results, CV-s, social and cultural activity. Profile each team member through the personality characteristics. The profiling should be a continous process, as people change over time. Enabling the process will give a better insight on the current and future actions of the team.
\newline\indent
Use agile methodologies, because are focused on communication and collaboration, which should easily level up the distance and knowledge between the team components.
\newline\indent
Organize trainings held by both of the two parts - the young will give a training on a newly appeared technology and the seniors on existing technologies that they master perfectly
\newline\indent
Organize a teambuilding once 6 sprints, considering a sprint of 2 weeks, so that people get to a better understanding of also the social, moral values of the other team members.
\newline\indent
Associate a younger developer a senior developer as a supervisor and pillar of knowledge. When the junior is in neeed of help, the senior will be the first stop when seeking help.
\newline\indent
Allocate to the junior developers 20\% percent of the time spent at work to learn new things, without any pressure from the leader or management side.
\newline\indent
Involve employees to set their own goals, as a team.
\newline\indent
Security as a need - always have a peer on whom he can count on, a junior can feeel more secure. It will achieve predictability by delegation. Although the peer may not be in the position to always solve the junior's problem or to direct it in a right direction, it will suffice by giving more confidence.(page 43)
\newline\indent
Acknowledge and address the danger - unwanted behavior, of undesired unofficial groups being created or alliances between the team members of same age. In the same time, people are social and gregarious beings, so group formation must not be interdicted or suppressed, only controlled and directed to organization's and project goals.
\newline\indent
Competence - break things and then try to put the things together again. It will inspire the juniors and they will believe more in their capabilities and power. Achievement oriented developers. Being a producer doesn't mean you are a good manager, as people are not always full-time producers. (page 50)
\newline\indent
People can respond in responsible and productive ways to a work environment in which they are given an opportunity to grow and mature. People begin to satisfy their esteem and self-actualization needs by participating in the planning, organizing, motivating and controlling of their own tasks.(page 72)
\newline\indent
Problems: the juniors may not have so many personal constraints like children, marriage, long-time relationships in private life. Seniors may be reluctant and not so willing to spend their entire creative energy on job-related issues. The familly may be, for them, on the first plan, unlike the juniors, who may put it first.
\section{Ideas}
The project manager has a fulltime job.
Situational leadership.
Create value as perceived by followers.
Power changes from seller to buyers - Kickstart/Indiegogo.
Technology is getting borders closer.
What can be done will be done, sooner or later, if not by you, by some one else.
Real leaders believe in change.
Behavioural sciences deal with probabilities.
Managers do things right. Leaders do right things.
Performance starts from bottom-up, with the one-to-one relationships.
Are leaders born or made?
Diagnose, adapt and communicate.
Plan, organize, motivate and control.
Understand, predict and change behaviour. Control means manipulation and viceversa, depending on the point of view and the expected outcome. :) uuf, I am so relieved.
A hammer won't always do the job.
If you want to change your behaviour, you have to practice. Reading/learning will give you just a conceptual view. :)) At last, someone agrees with me.\cite{IEEEhowto:fulltime}\\
A goal that it is appropriated for a 6-year-old may not be a meaningful goal for a 7-year-old. Coworkers can participate in setting their own goals.
Keep the carrot in the donkey's reach.(page 32)
\\
\indent
Although \emph{David W. Galenson}, in his work \emph{``Old Masters and Young Geniuses: The Two Life Cycles of Artistic Creativity"}, refers to plastic artists, I strongly believe that the same ideas can be applied to programming and software engineering. As a consequence, software products can be considered as the unique resut of an art form. The team members' creativity and past knowledge can balance each other. That if the process is managed properly, with the leader acting as a middleman between the two molding forces - innovation and reuse. The ability to achieve a goal, to finish successfully a task can be native or hard-earned. One may innovate successfully at the conceptual level, but it requires a long time of trial-and-error or learning for experimental and solid innovation. The ability to radically innovate can decrease with age, as a person is inclined to self-establish a set of rigid methods and conventions.\cite{IEEEhowto:oldvsyoung}

\epigraph{``When a situation requires a new way of looking at things, the acquisition of new techniques, or even new vocabularies, the old seem stereotyped and rigid....But when a situation requires a store of past knowledge then the old find their advantage over the young."}{--- \textup{Harvey Lehman}, 1953
}

\newpage
\section{Juniors}
\section{Seniors}

\newpage
\section{Conclusion}

% use section* for acknowledgement
\section*{Acknowledgment}
The author would like to thank to his teachers.

\newpage
\begin{thebibliography}{99}


\bibitem{IEEEhowto:fulltime}
Paul Hersey, Kenneth H. Blanchard and Dewey E. Johnson, \emph{Management of Organizational Behavior: Leading Human Resources}

\bibitem{IEEEhowto:oldvsyoung}
David W. Galenson, \emph{Old Masters and Young Geniuses: The Two Life Cycles of Artistic Creativity}

\bibitem{IEEEhowto:lehman}
\emph{Harvey Lehman, 1953} 

\end{thebibliography}

\end{document}


